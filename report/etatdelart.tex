\chapter{Related Work}
\label{chap:related-work}

Genomics is a field that has been growing rapidly in the past few years. The advent of high-throughput
sequencing technologies has made it possible to sequence the entire genome of an organism in a matter
of days. This has led to an explosion of data, with the number of sequenced genomes increasing exponentially.
This has created a need for new tools and algorithms to analyze this data. In this chapter, we review 
some of the existing tools and algorithms for analyzing genomic data.

\section{Genomic data analysis}
\label{sec:genomic-data-analysis}

\subsection*{Random forests}
\label{subsec:random-forests}

Another method that has been used for genomic data analysis is random forests (RF) \cite{Chen-Ishwaran-2012}.
This method is based on the idea of ensemble learning, where multiple decision trees are trained on different
subsets of the data and then combined to make a final prediction.

\subsection*{AI applications in genomic analysis}
\label{sec:ai-applications-in-genomic-analysis}

Many researchers have used AI techniques to analyze genomic data. For example, \cite{Caudai-et-al-2021} reviews
different AI techniques that have been used for genomic data analysis, including CNNs, autoencoders, etc.

More recently, the autors of \cite{Zhou-et-al-2024} designed a tool based on multiple LLM backends for multi-omics
analysis with minimal human intervention.
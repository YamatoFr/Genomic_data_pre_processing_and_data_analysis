\chapter{Introduction}
\label{chap:introduction}

Tuberculosis (TB) is a highly infectious disease caused by the bacterium \textit{Mycobacterium Tuberculosis}.
It primarily affects the lungs but can also impact other parts of the body. TB is a major global health concern,
with millions of new cases and deaths reported annually. The disease is transmitted through airborne droplets
when an infected person coughs or sneezes.

The emergence of multidrug-resistant (MDR) and extensively drug-resistant (XDR)\\strains of \textit{M. tuberculosis}
poses a significant challenge to TB control efforts. These strains are resistant to the most effective anti-TB drugs,
making treatment more difficult and increasing the risk of transmission. The World Health Organization (WHO) has
identified the need for new diagnostic tools, treatments, and vaccines to combat TB, particularly in the context of
drug resistance.

This project aims to find a way to identify the strain of \textit{M. tuberculosis} responsible for a TB infection using
genomic data. By analyzing the genetic sequences of \textit{M. tuberculosis} strains, we hope to develop a method that can
accurately determine the strain type, which is crucial for effective treatment and control of the disease. We explore
different approaches to achieve this goal.

We first use multiple clustering techniques to group similar strains based on the presence or absence of certain proteins.
Then, we converted the genomic data into images of various resolutions and used a convolutional neural network (CNN) to
classify the images.

\addcontentsline{toc}{chapter}{abstract}

\hspace{0pt}
\vfill
\begin{otherlanguage}{french}
	\begin{abstract}
		\begin{otherlanguage}{french}
			La tuberculose (TB), causée par Mycobacterium tuberculosis, reste l'une des maladies infectieuses les plus meurtrières
			dans le monde. Comprendre les différentes souches est essentiel pour améliorer les traitements et les stratégies de lutte.
			Ce projet vise à identifier les souches de TB à partir de données de séquençage du génome complet. Nous avons exploré
			des méthodes de regroupement non supervisées et développé des modèles de classification à l'aide de réseaux de neurones
			convolutifs (CNN). Des techniques comme les autoencodeurs et l'augmentation de données ont été utilisées pour améliorer
			les performances. Bien que les résultats soient encore préliminaires, ils montrent un potentiel pour différencier les
			souches de TB selon leurs profils génomiques, ouvrant la voie à des outils de diagnostic plus précis.
		\end{otherlanguage}
	\end{abstract}
\end{otherlanguage}

\begin{abstract}
	Tuberculosis (TB), caused by Mycobacterium tuberculosis, remains one of the\\world's most deadly infectious diseases. Understanding
	its different strain types is vital for improving treatment and control efforts. This project focuses on identifying TB strains
	using whole genome sequencing data. We explored unsupervised clustering methods and developed classification models using
	convolutional neural networks (CNNs). Techniques such as autoencoders and data augmentation were also applied to enhance
	performance. While the results are preliminary, they demonstrate potential for differentiating TB strains based on genomic
	patterns, laying the groundwork for more refined diagnostic tools.
\end{abstract}
\vfill
\hspace{0pt}
\chapter{Conclusion}
\label{chap:conclusion}

This project focused on identifying strain types of Mycobacterium tuberculosis using whole genome sequencing data.
We applied clustering techniques and convolutional neural networks to analyze genomic patterns, with the goal of
developing an effective classification method. While the results are still preliminary, they show potential for
distinguishing TB strains and suggest that further work could lead to more reliable models.

The project also provided practical experience in applying machine learning to genomic data. It involved working
with real biological datasets, managing data preprocessing, and evaluating model performance. These tasks helped
me better understand the challenges of working in bioinformatics, especially when dealing with complex and noisy
data.

In addition to technical skills, the internship helped me improve how I approach research problems, organize
experiments, and interpret results. It confirmed my interest in the field and gave me a solid starting point
for future work in combining data science with biology.
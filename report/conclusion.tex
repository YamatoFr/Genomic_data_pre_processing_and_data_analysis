\chapter{Conclusion}
\label{chap:conclusion}

During this project, we have explored various techniques for genomic data pre-processing and analysis, focusing on the Mycobacterium tuberculosis (MTB)
genome. We have implemented several methods for data pre-processing and analysis, such as K-means clustering, DBSCAN, HDBSCAN and an autoencoder.
However, we encountered several challenges, particularly with the clustering methods, which did not yield satisfactory results. We theorized that,
either the data was not suitable for clustering due to its high dimensionality, or that they were not prepared correctly.

We also explored the use of Convolutional Neural Networks (CNNs) for image classification, which showed promising results. The images were generated
from the genomic data by applying a combination of three of five metrics to the nucleotide sequences, resulting in ten different combinations of metrics,
each represented by a different color channel in the image. We also created a dataset of mosaic images, combining the images from the ten different
combinations of metrics.

The resulting dataset was highly imbalanced, with a significant number of samples belonging to two of the fives classes,
Euro-American and East-Asian. To address this issue, we applied different techniques such as K-fold cross validation, under-sampling, data augmentation
using SMOTE, and a combination of both under-sampling and data augmentation. The best results were achieved with the data augmentation technique,
which improved the metrics for the Indo-Oceanic class on the individual images, and the East-African Indian class on the mosaic images.

Another behavior we observed was that, despite being the class with the lowest number of samples, the M class always had metrics on par or close to the
majority classes. This suggests that the M class has a specific pattern that is distinct from the other classes, which could be further investigated
in future work.

In conclusion, this project has provided valuable insights into the challenges and opportunities of genomic data pre-processing and analysis.
We have demonstrated the potential of using CNNs for image classification of genomic data, and the importance of addressing class imbalance
in the dataset. I personally learned a lot about the fields of genomics and bioinformatics. This project has been a great opportunity to apply
my knowledge in machine learning and data analysis to a real-world problem.